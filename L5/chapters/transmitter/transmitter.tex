The NavIC transmitter is simulated to send baseband signal to the channel as shown in Fig \ref{fig:trans_flow}. 

\begin{figure}[ht]
\centering
\includegraphics[width=1\columnwidth]{figs/trans_flow.jpg}
\centering
\captionsetup{justification=centering}
\caption{Transmitter Block diagram}
\label{fig:trans_flow}
\end{figure}


\section{Encoding}

\subsection{FEC Encoding}
The Navigation data subframe of $292$ bits is convolution encoded with a rate of 1/2 and clocked at $50$ symbols per second. The coding scheme is given in figure \ref{fig:FEC}. Each subframe of $292$ bits, after encoding, results in $584$ symbols. Various FEC encoding parameters used in NavIC are tabulated in table \ref{table:fec_enc}.
\begin{figure}[ht]
\centering
\includegraphics[width=\columnwidth]{figs/FEC.png}
\centering
\captionsetup{justification=centering}
\caption{FEC Encoding}
\label{fig:FEC}
\end{figure}

\begin{table}[h]
%\centering
\input{tables/enc_param.tex}
\vspace{3mm}
\caption{FEC encoding parameters}
\label{table:fec_enc}
\end{table}
\subsection{Interleaving}
Any burst errors during the data transmission can be corrected by interleaving. In matrix interleaving, input symbols are filled into a matrix column wise and read at the output row wise. This will spread the burst error, if any, during the transmission. For SPS, data is filled into matrix of size $73$ by $8$($73$ columns, $8$ rows).

\section{PRN codes for SPS}
PRN Codes selected for Standard Positioning System are similar to GPS C/A Gold codes. The length of each code is $1023$ chips. The code is chipped at $1.023$ Mcps.

\noindent For SPS code generation, the two polynomials G1 and G2 are as defined below:
\begin{align}
G1 &: X^{10}+X^{3} + 1\\
G2 &: X^{10}+X^9+X^8+X^6+X^3+X^2+1
\end{align}
Polynomial G1 and G2 are similar to the ones used by GPS C/A signal. The G1 and G2 generators are realized by using $10$ bits Maximum Length Feedback Shift Registers(MLFSR). The initial state of G2 provides the chip delay. The G1 register is initialized with all bits as $1$. G1 and G2 are XOR'ed for the generation of the final $1023$ chip long PRN sequence. The SPS PRN code generator is shown in figure \ref{figure:codeGen}.

\begin{figure}[!ht]
	\centering
	\input{figs/spsCode_genBlock.tikz}
	\caption{SPS PRN Code Generator}
	\label{figure:codeGen}
\end{figure}

\noindent The satellites that are used in the current simulation are with PRN ids - $1,3,5$ and $7$. The initial condition of G2 register in L5 and S bands for the given PRN ids is given in table \ref{table:sps_codes}.

\begin{table}[h]
%\centering
\input{tables/sps_codes.tex}
\vspace{3mm}
\caption{Code phase assignment for SPS signals}
\label{table:sps_codes}
\end{table}

\section{Modulation}

\subsection{Standard Positioning Service}
The SPS signal is BPSK(1) modulated on L5 and S bands. The navigation data at data rate of $50$ sps (1/2 rate FEC encoded) is modulo $2$ added to PRN code chipped at $1.023$ Mcps. The CDMA modulated code is upconverted by the L5 and S carriers at $1176.45$ MHz and $2492.028$ MHz respectively. However, this simulation does not carry out this upconversion.

\subsection{Baseband Modulation}
The carrier signal is modulated by BPSK(1), Data channel BOC(5,2), and Pilot Channel BOC(5,2). To have a constant envelop when passed through power amplifier, we add additional signal called interplex signal.

\subsubsection{Mathematical Equations}
The following equations describe baseband signals for L5 and S band navigation data.
\\
\textbf{SPS Data Signal:}
\begin{equation}
	s_{sps}(t) = \sum_{i=-\infty}^{\infty} c_{sps}(\abs{i}_{L\_sps}) . d_{sps}(\sbrak{i}_{CD\_sps}) . \rect_{T_{c,sps}}(t-iT_{c,sps})
\end{equation}
\textbf{RS BOC Pilot Signal:}
\begin{equation}
	s_{rs\_p}(t) = \sum_{i=-\infty}^{\infty} c_{rs\_p}(\abs{i}_{L\_rs\_p}) . \rect_{Tc,rs\_p}(t-iT_{c,rs\_p}). sc_{rs\_p}(t,0)
\end{equation}
\textbf{RS BOC Signal:}
%\begin{equation}
\begin{multline}
	s_{rs\_d}(t) = \sum_{i=-\infty}^{\infty} c_{rs\_d}(\abs{i}_{L\_rs\_d}) . d_{rs\_d}(\sbrak{i}_{CD\_rs\_d}).  \\ 
	                       \rect_{T_{c,rs\_d}}(t-iT_{c,rs\_d}). sc_{rs_d}(t,0)
\end{multline}
%\end{equation}
The sub-carrier is defined as:
\begin{equation}
	sc_x(t,\phi) = \sgn\sbrak{sin(2 \pi f_{sc,x}t + \phi)}
\end{equation}

\noindent The IRNSS RS data and pilot BOC signals are sinBOC. Hence the subcarrier phase $\phi=0$.
The complex envelope of composite signal with Interplex signal (I(t)) is:
\begin{equation}
s(t) = \dfrac{1}{3} \sbrak{\sqrt{2} (s_{sps}(t) + s_{rs\_p}(t)) + j(2. s_{rs\_d}(t) - I(t))} 
\end{equation}

\noindent The Interplex signal $I(t)$ is generated to realize the constant envelope composite signal. The operation $\abs{i}_X$ gives the code chip index for any signal. Similarly $[i]_X$ gives data bit index for any signal.
Symbol definitions are given in below table \ref{table:symbdesc}.

\begin{table}[h]
%\centering
\input{tables/table1.tex}
\vspace{3mm}
\caption{Symbol Description}
\label{table:symbdesc}
\end{table}

The functions for data generation, SPS-PRN sequence generation and baseband modulation are present in the below code.
\begin{lstlisting}
    codes/transmitter/transmitter.py
\end{lstlisting}
\let\cleardoublepage\clearpage
