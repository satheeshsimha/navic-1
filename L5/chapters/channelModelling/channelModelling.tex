The phenomena modelled in the satellite communication channel are 
\begin{enumerate}
   \item Doppler shift
   \item Delay 
   \item Power scaling and 
   \item Thermal noise at the receiver
\end{enumerate}

\section{Doppler shift}
Due to relative motion between the satellites and the receiver, the transmitted signals undergo a frequency shift before arriving at the receiver. This shift %
in frequency is called Doppler shift and can be computed as
\begin{equation}
    f_{shift} = f_{d}-f_{c} = \brak{\frac{V_{rel}}{c-V_{S,dir}}}f_{c}  
\end{equation}
where,

$f_{shift}$ = Frequency shift due to Doppler effect

$f_{d}$ = Frequency observed at receiver

$f_{c}$ = Carrier frequency at transmitter

$V_{rel}$ = Relative velocity of transmitter and receiver

$V_{S,dir}$ = Velocity of satellite along radial direction

$c$ = Speed of light
\\

$V_{rel}$ is given by
\begin{align}
    V_{rel} &= V_{S,dir} - V_{R,dir}
\end{align}
where,

$V_{R,dir}$ = Velocity of receiver along radial direction

$V_{R,dir}$ and $V_{S,dir}$ are given by
\begin{align}
    V_{R,dir} &= \vec{V}_{R} \cdot \hat{\vec{dir}}\\
    V_{D,dir} &= \vec{V}_{S} \cdot \hat{\vec{dir}}
\end{align}
where,

$\hat{\vec{dir}}$ = Unit vector from satellite to receiver i.e. radial direction

$\vec{V_{S}}$ = Velocity of satellite

$\vec{V_{R}}$ = Velocity of receiver

$\hat{\vec{dir}}$ is given by
\begin{align}
    \hat{\vec{dir}} = \frac{\vec{P_{S}}-\vec{P_{R}}}{\norm{\vec{P_{S}}-\vec{P_{R}}}}
\end{align}
where,

$\vec{P_{S}}$ = Position of satellite

$\vec{P_{R}}$ = Position of receiver


The Doppler shift is introduced by muliplying the satellite signal with a complex exponential,
\begin{equation}
    x_{Shift}\sbrak{n} = x\sbrak{n}e^{-2 \pi j \brak{f_{c}+f_{Shift}} n t_{s}}
\end{equation}
where,

$x_{Shift}\sbrak{n}$ = Doppler shifted signal

$x\sbrak{n}$ = Satellite signal

$t_{s}$ = Sampling period

\section{Delay}
Since there is a finite distance between the satellite and the receiver, the signal at the reciever is a delayed version of the transmitted signal. This delay is given by
\begin{equation}
    D_{s} = \frac{d}{c}f_{s} 
\end{equation}
where,

$D_{s}$ = Total delay in samples

$d$ = Distance between satellite and receiver

$c$ = Speed of light

$f_{s}$ = Sampling rate

The total delay on the satellite signal is modeled in two steps. First, a static delay is modeled which does not change with time and it is always an integer number of samples. Then, %
a variable delay is modeled which can be a rational number of samples. While modelling the static delay, the entire delay is not introduced so that variable delay modelling handles the remaining %
delay.

To introduce the static delay, the samples are read from a queue whose size is the desired static delay length. When samples are read from the queue, an equal number of new samples are %
updated in the queue. To introduce the variable delay, the signal is passed through an all-pass FIR filter with an almost constant phase response. Its coefficients are calculated %
using the delay value required.

\section{Power Scaling}
When a transmitting antenna transmits radio waves to a receiving antenna, the radio wave power received is given by,
\begin{equation}
    P_r = P_t D_t D_r \brak{\frac{1}{4 \pi \brak{f_c + f_{Shift}} D}}^2
\end{equation}
where,

$P_r$ = Received power

$P_t$ = Transmitted power

$D_t$ = Directivity of transmitting antenna 

$D_r$ = Directivity of receiving antenna 

$D$ = Total delay in seconds

To scale the received signal as per the received power calculated,
\begin{equation}
    x_{Scaled}\sbrak{n} = \frac{\sqrt{P_r}}{\operatorname{rms}\brak{x\sbrak{n}}}x\sbrak{n}
\end{equation}   

\section{Thermal noise}
The thermal noise power at the receiver is given by,
\begin{equation}
    N_r = k T B
\end{equation}
where,

$N_r$ = Noise power in watts

$k$ = Boltzmann's constant

$T$ = Temperature in Kelvin

$B$ = Bandwidth in Hz

AWGN (Additive White Guassian Noise) samples with zero mean and variance $N_r$ are generated and added to the satellite signal to model thermal noise at receiver.

The functions necessary to model the channel are present in the below code,
\begin{lstlisting}
    codes/channelmodel/channelmodel.py
\end{lstlisting}
\let\cleardoublepage\clearpage
