\begin{figure}
\includegraphics[scale=0.5]{figs/block1}
\caption{The Block Level Architecture}
\label{fig:Block-diagram}
\end{figure}
\section{Hardware Implementation}
\begin{figure}
\includegraphics[scale=0.2]{figs/hardwareimplement.jpg} 
\caption{Reciever Hardware implementation}
\label{fig:rtl-sdr}
\end{figure}
List the various components used to implement receiver.
\\
\solution
\\
\begin{table}[!ht]
  \centering
 \input{tables/cmptable.tex}
  \caption{Components Required}
  \label{tab:rxcomponents}
\end{table}
Components are listed in the table \tabref{tab:rxcomponents}\\
The picture of RTL-SDR and Antenna is given in \figref{fig:rtl-sdr}.This set is used to receive the L5,L1,S Band  signals.
\begin{figure}[H]
\centering
\includegraphics[width=0.5\columnwidth]{figs/rtl-sdr.png}
\caption{RTL-SDR}
\label{fig:rtl-sdr}
\end{figure}
\section{RTL SDR Specifications}
RTL-SDR is a popular, low cost hardware that can receive wireless signals. The RTL-SDR dongle features the Realtek RTL2838U chip, which provides I-Q samples through the USB interface. 
\begin{table}[!ht]
  \centering
 \input{tables/sdrtable.tex}
  \caption{RTL-SDR Specification table }
  \label{tab:rxcomponents}
\end{table}

 Install and open GNU Radio using the following commands
\\
\begin{lstlisting}
sudo apt update
sudo apt install gnuradio
gnuradio-companion
\end{lstlisting}
 How to construct the block diagram in GNU radio? \\
	\solution  \\
\textbf{Step-1}:\\
Search for QTGUI Time sink  block and add it to the work space.
\begin{figure}[H]
\centering
\includegraphics[width=\columnwidth]{figs/add.png}
\caption{Adding blocks}
\label{fig:add blocks}
\end{figure}
\textbf{Step-2}:
Similarly do for RTL-Source block,complex to real ,complex to imaginary and QTGUI Time sink block.
\\
\textbf{Step-3}:
Change the parameters in each block according your values by double clicking on it.
\\
\textbf{Step-4}:
Connect them according to the flowgraph shown in \figref{fig:Rx_Block_diagram}.
\begin{figure}[H]
\centering
\includegraphics[width=\columnwidth]{figs/RTL_SDR_test.png}
\caption{Block diagram of Receiver in GNU Radio}
\label{fig:Rx_Block_diagram}
\end{figure}
\textbf{Note}:
Refer the following website for any queries.
\begin{lstlisting}
https://wiki.gnuradio.org/index.php?title=Creating_Your_First_Block
\end{lstlisting}
 Explain each block in block diagram of Receiver.\\
	\solution \\
\textbf{1. RTL-SDR Source}:\\
The RTL-SDR Source block is used to stream samples from RTL-SDR device.
\begin{figure}[H]
\centering
\includegraphics[width=0.4\columnwidth]{figs/source_block.png}
\caption{RTL-SDR Source Block}
\label{fig:source block}
\end{figure}
Connect the RTL-SDR to the system and execute the flowgraph in \figref{fig:Rx_Block_diagram}.\\



\subsection{Results} 
\begin{figure}
\includegraphics[width=0.8\columnwidth]{figs/RTL_sdr_res.png}
\caption{RTL-SDR receiver plots}
\end{figure}
I,Q samples will be stored in a file with .bin extension

 Read the data from RTL SDR stored in .bin file using python code and without using GNU radio.
\\
\solution \\
The following code reads the data from RTL-SDR without GNU radio.
\begin{lstlisting}
codes/receiver/rtsdr_rx.py
\end{lstlisting}














\section{USRP SDR}
The picture of Universal software Radio Peripheral-software defiend radios (USRP-SDR)  is given in \figref{fig:USRP-SDR}.This set is used to receive the L5,L1,S Band  signals.
\begin{figure}[H]
\centering
\includegraphics[width=0.5\columnwidth]{figs/USRP.png}
\caption{USRP-SDR}
\label{fig:USRP-SDR}
\end{figure}

\subsection{ USRP SDR specification}
 \begin{table}[!ht]
  \centering
 \input{tables/usrptable.tex}
 \caption{USRP-SDR Specification table }
\end{table}

 Install and open GNU Radio using the following commands
\\
\begin{lstlisting}
sudo apt update
sudo apt install gnuradio
gnuradio-companion
\end{lstlisting}
 How to construct the block diagram in GNU radio? \\
	\solution  \\
\textbf{Step-1}:\\
Search for QTGUI Time sink  block and add it to the work space.
\begin{figure}[H]
\centering
\includegraphics[width=\columnwidth]{figs/add.png}
\caption{Adding blocks}
\label{fig:add blocks}
\end{figure}
\textbf{Step-2}:
Similarly do for USRP-Source block,complex to real ,complex to imaginary and QTGUI Time sink block.
\\
\textbf{Step-3}:
Change the parameters in each block according your values by double clicking on it.
\\
\textbf{Step-4}:
Connect them according to the flowgraph shown in \figref{fig:Rx_Block_diagram}.
\begin{figure}[H]
\centering
\includegraphics[width=\columnwidth]{figs/USRP_navic.jpg}
\caption{Block diagram of Receiver in GNU Radio}
\label{fig:Rx_Block_diagram}
\end{figure}
\textbf{Note}:
Refer the following website for any queries.
\begin{lstlisting}
https://wiki.gnuradio.org/index.php?title=Creating_Your_First_Block
\end{lstlisting}
 Explain each block in block diagram of Receiver.\\
	\solution \\
\textbf{1. RTL-SDR Source}:\\
The RTL-SDR Source block is used to stream samples from USRP-SDR device.
\begin{figure}[H]
\centering
\includegraphics[width=0.4\columnwidth]{figs/usrp-sink.jpg}
\caption{USRP-SDR Source Block}
\label{fig:source block}
\end{figure}
Connect the USRP-SDR as shown flowgraph in \figref{fig:Rx_Block_diagram}.\\



\subsection{Results} 
\begin{figure}
\includegraphics[width=0.8\columnwidth]{figs/USRP_results.jpg}
\caption{USRP receiver plots}
\label{fig:plots}
\end{figure}

I,Q samples will be stored in a file with .bin extension

Read the data from RTL SDR stored in .bin file using python code and without using GNU radio.
\\
\solution \\
The following code reads the data from RTL-SDR without GNU radio.
\begin{lstlisting}
codes/receiver/usrp_rx.py
\end{lstlisting}
















