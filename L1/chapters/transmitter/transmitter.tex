The NavIC transmitter is simulated to send baseband signal to the channel as shown in Fig \ref{fig:trans_flow}. 

\begin{figure}[ht]
\centering
\includegraphics[width=1\columnwidth]{figs/trans_flow.jpg}
\centering
\captionsetup{justification=centering}
\caption{Transmitter Block diagram}
\label{fig:trans_flow}
\end{figure}

\section{Encoding}

\subsection{BCH Coding}

To transmit nine bits of Time of Interval (TOI) data, we employ BCH $(52, 9)$ coding. The generator polynomial used in this encoding process is $1767$ in octal notation. This coding scheme is illustrated conceptually in Figure~\ref{fig:generator} below, utilizing a 9-stage linear shift register generator.

\begin{figure}[ht]
    \centering
    \includegraphics[width=1.05\columnwidth]{figs/bch.png}
    \caption{Diagram of the BCH Encoder Circuit}
    \label{fig:generator}
\end{figure}

\noindent\textbf{Encoding Process}

\noindent Load TOI data bits $1$ to $9$ into the generator, starting with the Most Significant Bit (MSB).Shift the loaded data $52$ times through the generator to generate $52$ encoded symbols.The BCH $(n, k)$ encoders are realized using $k$-stage registers as illustrated in Figure~\ref{fig:generator}. During encoding, Gate $1$ is closed for the initial $k$ clock periods and then disconnected. Likewise, Gate $2$ is disconnected during the first $k$ periods and then closed.



\begin{table}[h]
\centering
\caption{Generator Polynomials of BCH Encoders}
\label{table:generator_polynomials}
\begin{tabular}{|l|lll|l|}
\hline
\multirow{2}{*}{BCH Code} & \multicolumn{3}{c|}{Encoding Characteristics} & \multirow{2}{*}{Generator Polynomials (g(x))} \\
\cline{2-4}
& $n$ & $k$ & $d_{\text{min}}$ & \\
\hline
(52, 9) & 52 & 9 & 20 & $x^9 + x^8 + x^7 + x^6 + x^5 + x^4 + x^2 + x + 1$ \\
\hline
\end{tabular}
\end{table}

\subsection{LDPC}


\noindent The LDPC (Low-Density Parity-Check) encoder structure is based on a parity-check matrix $H(m, n)$ consisting of $m$ rows and $n$ columns. Specifically, for subframe-2, $m = 600$ and $n = 1200$, and for subframe-3, $m = 274$ and $n = 548$ are selected.

\noindent The LDPC matrix $H$ is assumed to be in an approximate lower triangular form with a dual diagonal structure. Matrix $H(m, n)$ is further decomposed into six submatrices: $A$, $B$, $T$, $C$, $D$, and $E$, as illustrated in Figure~\ref{fig:ldpc-structure} below.

\begin{figure}[ht]
\centering
\includegraphics[width=1\columnwidth]{figs/ldpc.png}    
\caption{LDPC Matrix Structure}
\label{fig:ldpc-structure}
\end{figure}

\noindent Each element of matrix $H(m, n)$ takes on either the value $"0"$ or $"1"$.The inverse of matrix $T$, denoted as $T^{-1}$, is not included in this document. However, it is worth noting that since $T$ is a lower triangular matrix, its inverse can be readily identified.

For a rate $1/2$ LDPC encoder, the encoding process utilizes the matrices $A$, $B$, $T$, $C$, $D$, and $E$ to generate the encoded symbols based on the following algorithm:
\begin{align}
p_1^t &= -\varphi^{-1} (-E \cdot T^{-1} \cdot A + C) \cdot s^t \\
p_2^t &= -T^{-1} (A \cdot s^t + B \cdot p_1^t)
\end{align}


\noindent Where:
\begin{align*}
\varphi &= -E \cdot T^{-1} \cdot B + D \\
s &= \text{subframe 2 and subframe 3 data} \\
x_t &\text{ indicates transpose} \\
\end{align*}

\noindent The elements of matrices $p_1$ and $p_2$ are modulo 2 numbers.

\noindent The encoded symbols for broadcast are composed of $(s, p_1, p_2)$, where $s$ represents the systematic portion of the codeword, and $\{p_1, p_2\}$ constitute the combined parity bits.


\subsection{Interleaving}
\noindent Any burst errors during the data transmission can be corrected by interleaving. In matrix interleaving, input symbols are filled into a matrix column-wise and read at the output row-wise. This will spread the burst error, if any, during the transmission.The 1748 symbols of LDPC encoded navigation data of subframe-2 and subframe-3 are interleaved using a block interleaver with $n$ columns and $k$ rows. Data is written in columns and then read in rows. The Table \ref{tab:interleaving} below indicates the interleaving mechanism.

\begin{table}[ht]
\centering
\begin{tabular}{|l|l|}
\hline
\textbf{Parameter} & \textbf{Arrangement} \\
\hline
Block Interleaver size & 1748 \\
\hline
Block Interleaver Dimensions (n columns x k rows) & 46 x 38 \\
\hline
\end{tabular}
\caption{Interleaving Parameters}
\label{tab:interleaving}
\end{table}



\subsection{PRN codes for SPS}

\noindent The NavIC L1 signal utilizes a family of Interleaved Z4 – Linear (IZ4) PRN spreading codes implemented using coupled shift registers. The PRN code has a length of 10230 chips with a code period of 10 ms in both data and pilot channels. Furthermore, the pilot channel incorporates a secondary overlay code with a length of 1800 and a period of 18 s. Importantly, the pilot and data signals are designed to be orthogonal.
The IZ4 family of spreading codes has been found to deliver superior or equivalent performance compared to the PRN code families employed by GPS and BeiDou in the L1 band. Moreover, the resources required for implementing the code generator are of the same order as Weil codes.


\begin{figure}[ht]
\centering
\includegraphics[width=\columnwidth]{figs/tiered_code}
\centering
\captionsetup{justification=centering}
\caption{Tiered code structure and timing relationship between primary and secondary codes}
\label{fig:R0_IZ4}
\end{figure}



\begin{table}[h]
%\centering
\small
% Contents of table1.tex
%\begin{tabular}{|c|c|c|c|c|c|c|}
\begin{tabular}{|c|p{1.5cm}|p{1.3cm}|p{1.5cm}|p{1.5cm}|p{1.5cm}|p{1.5cm}|}
\hline
\textbf{Component} & \textbf{Primary Code Type} & \textbf{Primary Code Length} & \textbf{Primary Code Period (msec)} & \textbf{Secondary or Overlay Code Type} & \textbf{Secondary or Overlay Code Length} & \textbf{Secondary or Overlay Code Period (msec)} \\
\hline
L1 Data (L1D i (t)) & Z4–linear (IZ4) & 10230 & 10 & - & - & - \\
\hline
L1 Pilot (L1P i (t)) & Interleaved Z4–linear (IZ4) & 10230 & 10 & Truncated Z4-linear sequences & 1800 & 18000 \\
\hline
\end{tabular}

\vspace{3mm}
\caption{Characteristics of the L1 ranging codes}
\label{table:L1_ranging}
\end{table}

\subsubsection{Code Generator Architecture for primary L1 Pilot and L1 Data PRN Code}

\noindent The IZ4 ranging code generator comprises the following principal components:

\begin{enumerate}
    \item {Shift Registers R0 and R1:} These are two fifty-five tap binary shift registers.
    \item {Shift Register C:} This is a single, five-tap, binary, pure-cycling shift register.
\end{enumerate}

\textbf{Code Generation}

\noindent The IZ4 code is generated as the chip-by-chip modulo-2 sum of the synchronized output of Shift Register C and Register R1. Specifically, it is computed using the following equation:
\begin{align}
IZ4(t) = C(t, 0) \oplus R1(t, 0)
\label{eq:iz4} 
\end{align}
\noindent In this equation $C(t, 0)$ represents the first tap of Register C at time $t$, abbreviated as $C(0)$.
$R1(t, 0)$ represents the contents of the first tap of Register R1 at time $t$, abbreviated as $R1(0)$.
It's important to note that all three registers are synchronized with respect to each other, ensuring proper code generation.
It's important to note that all three registers are synchronized with respect to each other, ensuring proper code generation.

\begin{figure}[ht]
\centering
\includegraphics[width=0.8\columnwidth]{figs/IZ4_BCH.png}
\centering
\captionsetup{justification=centering}
\caption{Functional Description of IZ4 Sequence generation using Binary shift registers.}
\label{fig:IZ4_BCH}
\end{figure}

\noindent Functional description of each register involved in the IZ4 code generation process:

\subsubsection{Shift Register R0}

\noindent A fifty-five tap long $R0$ shift register is the first component.This register generates binary codes with a period of $10230$, shifting its contents at each clock cycle. The register's initial state is determined by stored initial conditions. It produces a binary code sequence with a $10230$-chip period, resetting after 10230 cycles. Feedback operations are governed by a feedback polynomial, with the output fed back to the $55^{th}$ tap. The first tap,$ R0(0)$, provides the component code output. In this process, seven out of $55$ taps are employed, following the equation: 

\begin{equation}
R0(54) = R0(50) \oplus R0(45) \oplus R0(40) \oplus R0(20) \oplus R0(10) \oplus R0(5) \oplus R0(0) 
\label{eq:R0(54)} 
\end{equation}

\begin{figure}[ht]
\centering
\includegraphics[width=0.8\columnwidth]{figs/R0_IZ4.png}
\centering
\captionsetup{justification=centering}
\caption{Feedback logic used to generate the linear feedback to Register R0 for IZ4 codes}
\label{fig:R0_IZ4}
\end{figure}
\noindent The feedback logic, illustrating inputs representing tap contents at time $(t)$ and output corresponding to the 55th tap at $(t+1)$, is depicted in Figure \ref{fig:R0_IZ4}.

\subsubsection{Shift Register R1}

\noindent The shift register \(R1\), consisting of fifty-five taps, serves as the source for generating the second component of the IZ4 ranging code. Similar to \(R0\), it operates by producing a binary code with a period of 10230 through content shifts at each clock cycle. The initial state of \(R1\) is determined by stored initial conditions, and it resets after 10230 clock cycles. The output originates from the first tap.

\noindent Feedback to \(R1\) encompasses both \(R1A\) and \(R1B\) components, computed as functions of the taps from both \(R0\) and \(R1\). The feedback to \(R1\) is the modulo-2 sum of \(R1A\) and \(R1B\).

Specifically, three sub-components, \(\sigma_2A\), \(\sigma_2B\), and \(\sigma_2C\), which depend on the tap contents of \(R0\), contribute to the computation of \(R1A\) as per the following equations:

\begin{equation}
\sigma_2A = [R0(50) \oplus R0(45) \oplus R0(40)] \text{ AND } [R0(20) \oplus R0(10) \oplus R0(5) \oplus R0(0)] 
\label{eq:sigma_2A } 
\end{equation}
\begin{equation}
\sigma_2B = ([R0(50) \oplus R0(45)] \text{ AND } R0(40)) \oplus  ([R0(20) \oplus R0(10)] \text{ AND }[R0(5) \oplus R0(0)]) 
\label{eq:sigma_2B } 
\end{equation}
\begin{equation}
\label{eq:sigma_2c} 
\sigma_2C = [R0(50) \text{ AND } R0(45)] \oplus ([R0(20) \text{ AND } \\ R0(10)]\oplus  [R0(5) \text{ AND }R0(0)])  
\end{equation}
\begin{equation}
\label{eq:sigma_2} 
\sigma_2 = \sigma_2A \oplus \sigma_2B \oplus \sigma_2C
\end{equation}
\begin{equation}
\label{eq:R1A} 
R1A = \sigma_2 \oplus [R0(40) \oplus R0(35) \oplus R0(30) \oplus R0(25) \oplus R0(15) \oplus R0(0)]
\end{equation}
\begin{equation}
\label{eq:R1B} 
R1B = R1(50) \oplus R1(45) \oplus R1(40) \oplus R1(20) \oplus R1(10) \oplus R1(5) \oplus R1(0)
\end{equation}
\begin{equation}
\label{eq:R154} 
R1(54) = R1A \oplus R1B
\end{equation}

In equations \ref{eq:sigma_2A } through \ref{eq:R1B}, all quantities represent contents or functions of register contents at time \(t\). Equation \ref{eq:R154} computes \(R1(54)\) at time \(t+1\) based on the quantities on the right side at time \(t\). All registers initialize with initial conditions at \(t=0\).
Implementation of the feedback computation for Shift Register \(R1\), where \(R1A\) and \(R1B\) are computed at time \(t\), and \(R1(54)\) refers to the contents \(R1(t+1,54)\) at time \(t+1\) of the 55th tap of Register \(R1\).

\begin{figure}[h]
    \centering
    \includegraphics[width=\columnwidth]{figs/fig8.png}
    \captionsetup{justification=centering}
    \caption{Feedback Computation for R0 and R1A Determination for Overlay Codes}
    \label{fig:R0overlay}
\end{figure}

\begin{figure}[h]
    \centering
    \includegraphics[width=\columnwidth]{figs/fig9.png}
    \captionsetup{justification=centering}
     \caption{$sigma2A$and $sigma2B$ Computation for R1A Determination for Primary IZ4 Codes}
    \label{fig:sigma2A_and_sigma2B_computation}
\end{figure}

\begin{figure}[h]
    \centering
    \includegraphics[width=\columnwidth]{figs/fig10.png}
    \captionsetup{justification=centering}
    \caption{$sigma2C$and $sigma2$ Computation for R1A Determination for Primary IZ4 Codes}
    \label{fig:sigma2C_and_sigma2_computation}
\end{figure}

\begin{figure}[h]
    \centering
    \includegraphics[width=\columnwidth]{figs/fig11.png}
    \captionsetup{justification=centering}
    \caption{R1 Feedback Computation Using R1 and R0 Shift Register Taps for Primary IZ4 Codes}
    \label{fig:R1Feedback}
\end{figure}





\subsubsection{Initial conditions for Shift Registers R0, R1 and C}
Each R0 and R1 component code generator uses 55 bits of unique initial conditions stored in memory.Shift Register C is a 5-tap pure-cycling register, where the first tap's output, denoted as C(0), is looped back as input to the fifth tap. C(0) is XORed with the output of the R1 shift register on a chip-by-chip basis to produce the IZ4 code. Initial five-bit conditions for Register C are provided in Tables 7 and 8, corresponding to each PRN code.
\subsection{Secondary Overlay Code Generator \\Architecture}
The secondary or overlay codes linked to each L1 pilot primary code have a length of 1800 chips. These overlay codes are independently synchronized in time and have a duration of 18 seconds, operating at a rate of 100 bps. The 1800-chip overlay codes are produced by cyclically cycling the Z4-linear codes, which have a period of 2046. The generation process for overlay codes resembles that of primary IZ4 codes. To generate overlay codes, two ten-tap shift registers, denoted as R0 and R1, are utilized with specific feedback polynomials. Unlike primary code generation, the C register is not necessary for the generation of overlay codes.

\begin{equation}
R0(9) = R0(5) \oplus R0(2) \oplus R0(1) \oplus R0(0) 
\end{equation}
\begin{equation}
\sigma_{2A} = [R0(5) \oplus R0(2)] \text{ AND } [R0(1) \oplus R0(0)]
\end{equation}
\begin{equation}
\sigma_{2B} = [R0(5) \text{ AND } R0(2)] \oplus [R0(1) \text{ AND } R0(0)] 
\end{equation}
\begin{equation}
R1A = \sigma_2 \oplus R0(6) \oplus R0(3) \oplus R0(2) \oplus R0(0) 
\end{equation}
\begin{equation}
R1B = R1(5) \oplus R1(2) \oplus R1(1) \oplus R1(0) 
\end{equation}
\begin{equation}
R1(9) = R1A \oplus R1B
\end{equation}


\begin{figure}[ht]
\centering
\includegraphics[width=\columnwidth]{figs/overlay.png}
\centering
\captionsetup{justification=centering}
\caption{Block Diagram of overlay sequence generator}
\label{fig:overlay}
\end{figure}

\begin{figure}[h]
    \centering
    \includegraphics[width=\columnwidth]{figs/fig13.png}
    \captionsetup{justification=centering}
    \caption{Feedback Computation for R0 and R1A Determination for Overlay Codes}
    \label{fig:R0overlay}
\end{figure}

\begin{figure}[h]
    \centering
    \includegraphics[width=\columnwidth]{figs/fig14.png}
    \captionsetup{justification=centering}
    \caption{$\sigma2A$ and $\sigma2B$ Computation for Overlay Codes}
    \label{fig:sigma2Boverlay}
\end{figure}

\begin{figure}[h]
    \centering
    \includegraphics[width=\columnwidth]{figs/fig15.png}
    \captionsetup{justification=centering}
    \caption{R1 Register Feedback Computation for Overlay Codes}
    \label{fig:R1overlay}
\end{figure}

In Figures \ref{fig:R0overlay},\ref{fig:sigma2Boverlay}, and \ref{fig:R1overlay} below, the feedback computation process for $R_0$ and $R_1$ shift registers for overlay codes is explained. Figure \ref{fig:R0overlay} demonstrates how the feedback for Register $R_0$ is performed, along with the computation of certain components of the feedback for Register $R_1$. The output of Register $R_1$ corresponds to the overlay code. Quantities shown on the left, such as $R_0(5)$ in the topmost sub-figure of Figure~\ref{fig:R0overlay}, represent the contents of the respective registers at time $t$, while the quantity $R_0(9)$ on the opposite side represents the contents of $R_0$ at time $t+1$ for position 9.
\newpage

\section{Modulation}

\subsection{Standard Positioning Service}
\noindent The SPS signal is modulated using Synthesized Binary Offset Carrier (SBOC) in L1 band and BPSK in L5 and  S bands.
\subsection{Baseband Modulation}
\noindent SBOC modulation contains  BOC(1,1) and BOC(6,1) components in both data signal  and pilot signals.  In this scheme, the data channel's BOC(1,1), the pilot channel's BOC(1,1), and the pilot channel's BOC(6,1) components are interplexed to create the data channel's BOC(6,1) component.  Data and pilot signals are quadrature multiplexed, with $41.82\%$ power to data and $58.18\%$ to pilot, ensuring constant envelope modulation for MBOC.
\subsubsection{Mathematical Equations}
\noindent The mathematical representation of baseband navigation signals is as follows:

\noindent\textbf{Pilot Signal:}
\begin{multline}
S_{p,a}(t) = \sum_{i=-\infty}^{\infty} C_{p,s}\Bigl[|i|_{1800}\Bigr] \oplus \sum_{j=1}^{10230}C_{p,p}\Bigl[j\Bigr]\cdot \\
             \text{rect}_{T_{c,p,p}} \left( t - iT_{c,p,s} - jT_{c,p,p}\right) \cdot sc_{p,a}(t, 0)
\label{eq:sp_a}
\end{multline}
\begin{multline}
S_{p,b}(t) =    \sum_{i=-\infty}^{\infty} C_{p,s}\Bigl[|i|_{1800}\Bigr] \oplus \sum_{j=1}^{10230}C_{p,p}\Bigl[j\Bigr]\cdot \\
    \text{rect}_{T_{c,p,p}} \left( t - iT_{c,p,s} - jT_{c,p,p}\right) \cdot sc_{p,b}(t, 0)
\label{eq:sp_b}
\end{multline}

\noindent where $C_{p,p}$ is pilot primary PRN code, $C_{p,s}$ is pilot secondary/overlay PRN code, 
$T_{c,p,p} = \frac{1}{1.023}\mu$s and $T_{c,p,s}= 10$ms. $|i|_{L}$ means i modulo L.\\

\noindent $S_{p,a}$ is sinBOC(1,1) component of pilot signal and $S_{p,b}$ is sinBOC(6,1) component 
of pilot signal.
\\

\noindent The Binary NRZ sub-carrier is defined as:

\begin{equation}
sc_{p,x}(t, \varphi) = \text{sgn}[\sin(2\pi f_{sc,x}t + \varphi)]
\label{eq:sub_carrier}
\end{equation}

\noindent The subcarrier signals are sinBOC. Hence, the subcarrier phase $\varphi=0$. \\
\noindent\textbf{Data Signal:}

\begin{multline}
S_{d,a}(t) = \sum_{i=-\infty}^{\infty} C_d\Bigl[|i|_{10230} \Bigr] \oplus d_d\Bigl[[i]_{10230}\Bigr] \cdot \\
\text{rect} _{T_{c,d}} \left({t - iT_{c,d}}\right) \cdot sc_{d,a}(t, 0)
\label{eq:signal_da}
\end{multline}
\noindent where $T_{c,d} = \frac{1}{1.023}\mu$s, $C_d$ is Data PRN code and $[i]_L$ means the integer part of $\frac{i}{L}$.\\

\noindent The interplexed component $S_{d,b}(t)$ is given by:
\begin{multline}
S_{d,b}(t) = \sum_{i=-\infty}^{\infty} C_d\Bigl[|i|_{10230}\Bigr] \oplus d_d\Bigl[[i]_{10230}\Bigr] \cdot \\
\text{rect}_{T_{c,d}} \left( t - iT_{c,d} \right) \cdot sc_{d,b}(t, 0)
\label{eq:interplexed_component}
\end{multline}

\noindent The above equation can also be represented as
\begin{equation}
    S_{d,b}(t) = S_{p,a}(t) \cdot S_{p,b}(t) \cdot S_{d,a}(t)
    \label{eq:interplexed_component1}
\end{equation}

\noindent The Binary NRZ sub-carrier is defined as:
\begin{equation}
sc_{d,x}(t, \varphi) = \text{sgn}[\sin(2\pi f_{sc,x}t + \varphi)]
\label{eq:subcarrier_dc}
\end{equation}

\noindent The subcarrier signals are sinBOC. Hence, the subcarrier phase $\phi=0$. \\

\noindent $f_{sc,a}$ is Sub-carrier frequency of $sc_{p,a}$ and $sc_{d,a}$ sub-carriers and equal to 1.023 MHz. 
$f_{sc,b}$ is Sub-carrier frequency of $sc_{p,b}$ and $sc_{d,b}$ sub-carriers and equal to 6.138 MHz. \\

\noindent Ranging code $C_d$, defined in \eqref{eq:signal_da} and \eqref{eq:interplexed_component}, includes only primary code of data signal. \\

\noindent The composite SBOC modulated signal $S(t)$  is generated by quadrature multiplexing of data and pilot signals, as given below:

\begin{equation}
S(t) = [\alpha S_{p,a}(t) - \beta S_{p,b}(t)] + j[\gamma S_{d,a}(t) + \eta S_{d,b}(t)]
\label{eq:composite_signal}
\end{equation}
\noindent where $\alpha = \sqrt{\frac{6}{11}}$, $\beta = \sqrt{\frac{4}{110}}$, $\gamma = \sqrt{\frac{4}{11}}$ and $\eta = \sqrt{\frac{6}{110}}$ \\


\begin{equation}
S(t) = S_I(t) + jS_Q(t)
\label{eq:baseband_composite}
\end{equation}

\noindent Based on \eqref{eq:baseband_composite}, the band-pass representation of the SBOC modulated navigation signal $(S_{RF}(t))$ at L1 band is defined as follows:

\begin{equation}
S_{RF}(t) = S_I(t) \cdot \cos(2\pi f_{L1} t) - S_Q(t) \cdot \sin(2\pi f_{L1} t)
\label{eq:bandpass_representation}
\end{equation}

\noindent where \(f_{L1}\) is equal to 1575.42 MHz.

\begin{table}[h]
%\centering
\input{tables/table1.tex}
\vspace{3mm}
\caption{Symbol Description}
\label{table:symbdesc}
\end{table}

\let\cleardoublepage\clearpage
